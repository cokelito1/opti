\documentclass[a4paper,oneside,10.5pt]{USMArt}

\usepackage{personal}
\usepackage{comment}
\usepackage[letterpaper]{geometry}

\title{Tarea I - Optimización y Control}
\sigla{MAT-379 }
\ramo{Optimización y Control}
\profesor{Patricio Guzman}
\semestre{2025-1}
\author{Jorge Bravo}

\begin{document}
\maketitle

\begin{sol}
  Sea $P = \{J(y) \; | \; y \in C\}$, donde $C = \{y \in \mathcal{C}^{1} \; | \; y(0) = 0, y(1) = 1\}$
  y

  \begin{equation*}
    J(y) = \int_{1}^{\frac12} |y(x)| dx + \int_{\frac12}^1 |y'(x)| dx
  \end{equation*}

  \begin{enumerate}
    \item Notemos que el conjunto C es no vació, pues la función $\text{Id}_{[0, 1]}$ cumple las condiciones para
      pertenecer a este. Luego podemos evaluar el funcional en $\text{Id}_{[0, 1]}$ y nos dará un numero pues estamos
      integrando una función continua sobre un compacto. Notemos que $|y(x)| \geq 0$ e $|y'(x)| \geq 0$
      para todo $y \in C$ y $x \in [0, 1]$, por lo tanto tenemos que $\forall y \in C, J(y) \geq 0$.
      Dado que mostramos que el conjunto $P$ es no vacío y tiene una cota inferior, existe $\inf P \in \RR$
      por axioma del supremo e ínfimo.

    \item Dado que en el paso anterior mostramos que $0$ es un cota inferior para el conjunto $P$ y el
     ínfimo es la mayor de las cotas superiores tenemos que

    \begin{equation*}
      0 \leq \inf P
    \end{equation*}

    \item La intuición nos dice que una función que minimiza $J$ que esta fuera del conjunto $C$
          es la indicatriz del conjunto $[\frac12, 1]$, construyamos una sucesión de funciones
          que la aproximen desde $C$. Consideremos la siguiente sucesión para $n \geq 3$

    \begin{equation*}
      f_{n}(x) = \begin{cases}
                   0 & \text{ si } x \in [0, \frac12 - \frac1n)\\
                   P_{n}(x) & \text{ si } x \in [\frac12 - \frac1n, \frac12]\\
                   1 & \text{ si } x \in (\frac12, 1]
                 \end{cases}
    \end{equation*}

    Donde $P_{n}$ sera un polinomio que a de cumplir con las siguientes condiciones
    \begin{align*}
      P_{n}(\frac12) &= 1\\
      P_{n}'(\frac12) &= 0\\
      P_{n}(\frac12 - \frac1n) &= 0\\
      P_{n}'(\frac12 - \frac1n) &= 0
    \end{align*}

    Pues si $P_{n}$ cumple con esas cuatro condiciones, entonces $f_{n} \in C$, dado que están
    de acuerdo en la derivada en los puntos de pegado y en el valor de la función también.
    Notemos que por construcción $f_{n}(0) = 0$ y $f_{n}(1) = 1$.

    Dado que tenemos $4$ condiciones sobre $P_{n}$, un polinomio de grado $3$ sera suficiente.
    Consideraremos $P_{n}(x) = a(x - \frac12)^{3} + b(x - \frac12)^{2} + c(x - \frac12) + d$.
    Luego las cuatro condiciones se transforman en lo siguiente

    \begin{align*}
       d &= 1\\
       c &= 0\\
       -\frac{a}{n^{3}} + \frac{b}{n^{2}} + 1 &= 0\\
       \frac{3a}{n^{2}} - \frac{2b}{n} &= 0
    \end{align*}

    Resolviendo el sistema llegamos al polinomio $P_{n}$, el cual por construcción satisface todo lo que necesitábamos.

    \begin{equation*}
      P_{n}(x) = -2n^{3}(x - \frac12)^{3} - 3n^{2}(x - \frac12)^{2} + 1
    \end{equation*}

    Luego $f_{n} \in C$. Probemos que en efecto $(f_{n})_{n \geq 3}$ es una sucesión minimizante.

    Notemos que

    \begin{align*}
      J(f_{n}) &= \int_{0}^{\frac{1}{2}} |f_{n}(x)| dx + \int_{\frac12}^{1} |f_{n}(x)'|dx\\
               &= \int_{0}^{\frac12 - \frac1n} 0 dx + \int_{\frac12 - \frac1n}^{\frac12} |P_{n}(x)| dx + \int_{\frac12}^{1} 0 dx\\
               &= \int_{\frac12 - \frac1n}^{\frac12} |P_{n}(x)| dx
    \end{align*}

    Dado que $P_{n}'$ es una cuadrática con coeficiente líder negativo, y sabemos
    que $P_{n}'(\frac12 - \frac1n) = 0 = P_{n}'(\frac12)$, tenemos
    que $\forall x \in [\frac12  - \frac1n, \frac12], P_{n}'(x) \geq 0$,
    es decir $P_{n}$ es creciente en ese intervalo, dado que $P_{n}(\frac12 - \frac1n) = 0$, tenemos que

    \begin{equation*}
      \forall x \in [\frac12 - \frac1n, \frac12], P_{n}(x) \geq 0
    \end{equation*}

    Luego seguimos con el calculo
    \begin{align*}
      J(f_{n}) &= \int_{\frac12 - \frac1n}^{\frac12} P_{n}(x) dx\\
               &= \int_{\frac12 - \frac1n}^{\frac12} -2n^{3}(x - \frac12)^{3} - 3n^{2}(x - \frac12)^{2} + 1 dx\\
               &= -\frac{1}{2}n^{3}(x - \frac12)^{4}|_{x = \frac12 - \frac1n}^{x = \frac12} - n^{2}(x - \frac12)^{3}|_{x = \frac12 - \frac1n}^{x = \frac12} + \frac1n\\
               &= \frac{1}{2n} - \frac{1}{n} + \frac{1}{n}\\
               &= \frac{1}{2n}
    \end{align*}

    Por lo tanto obtenemos que

    \begin{equation*}
      \lim_{n \to \infty} J(f_{n}) = 0
    \end{equation*}

    Es decir $(f_{n})_{n \geq 3}$ es una sucesión minimizante.

    \item No, no existe $\overline{y} \in C$ que minimice $J$, si suponemos que existe, entonces

    \begin{equation*}
      J(\overline{y}) = 0
    \end{equation*}

    Pues existe la sucesión minimizante a $0$ que construimos en el paso anterior y por tanto
    demostramos que $\inf P = 0$. Dado que $J(\overline{y})$ es la suma de dos cantidades positivas
    y tiene que ser igual a $0$, necesariamente cada una a de ser $0$.

    Notemos que

    \begin{equation*}
      \int_{0}^{\frac12} |\overline{y}(x)| dx = 0 \implies \forall x \in [0, \frac12], |\overline{y}(x)| = 0 \implies \forall x \in [0, \frac12], \overline{y}(x) = 0
    \end{equation*}

    Pues la cantidad de adentro es positiva y por tanto $0$ c.t.p. y por continuidad, en todas partes.
    Por lo tanto tenemos que $\overline{y}|_{[0, \frac12]} \equiv 0$.
    Análogamente, dado que estamos suponiendo $\overline{y} \in C$, $\overline{y}'$ es continua y por
    tanto su valor absoluto igual. Por el argumento anterior $\overline{y}'|_{[\frac12, 1]} \equiv 0$,
    por lo tanto $\overline{y}|_{[\frac12, 1]} \equiv c$ para algún $c \in \RR$,
    dado que $\overline{y}(1) = 1 \implies c = 1$.
    Esto es una contradicción pues entonces $0 = \overline{y}(\frac12) = 1$.
  \end{enumerate}
\end{sol}

\begin{sol}
  Usaremos el lema de Fermat. Supongamos que $y \in Y$ satisface el problema de minimización.
  Consideraremos una perturbación $h \in \mathcal{C}_{0}^{2}([0, L])$. Luego tenemos que

  \begin{equation*}
    \lim_{\varepsilon \to 0} \frac{J(y + \varepsilon h) - J(y)}{h} = \frac{d}{d \varepsilon} J(y + \varepsilon h)\big|_{\varepsilon = 0} = 0
  \end{equation*}

  Hagamos el calculo
  \begin{align*}
    \frac{d}{d \varepsilon} J(y + \varepsilon h) &= \frac{d}{d \varepsilon}(\frac12 \int_{0}^{L} EI(y''(x) + \varepsilon h''(x))^{2} dx - \int_{0}^{L} q(x)(y(x) + \varepsilon h(x)))\\
  \end{align*}

  Dada la regularidad de las funciones con las que estamos trabajando,
  podemos entrar la derivada dentro de las integrales

  \begin{align*}
    \frac{d}{d \varepsilon} J(y + \varepsilon h) &= \int_{0}^{L} EI(y''(x) + \varepsilon h''(x)) \cdot h''(x) dx - \int_{0}^{L} q(x)h(x) dx\\
                                                 &= \int_{0}^{L} -q(x)h(x) + EI(y''(x) + \varepsilon h''(x))h''(x) dx
  \end{align*}

  Luego tenemos que
  \begin{equation*}
    \int_{0}^{L} -q(x)h(x) + EIy''(x)h''(x) dx = \frac{d}{d \varepsilon} J(y + \varepsilon h)|_{\varepsilon = 0} = 0
  \end{equation*}

  Considerando $\alpha(x) = -q(x)$ y $\beta(x) = EIy''(x)$ podemos usar el lema,
  pues $q$ es continua por hipótesis e $y$ es continua pues estamos suponiendo que resuelve el problema. Por lo tanto
  concluimos que $EIy''(x) \in \mathcal{C}^{2}([0, L])$ lo que implica que $y \in \mathcal{C}^{4}([0, L])$. Tenemos
  entonces que
  \begin{equation*}
    y^{(4)}(x) = \frac{q(x)}{EI}
  \end{equation*}

  Luego dado que la viga esta apoyada en el punto $0$ y $L$, la variación angular en los puntos de contacto de la
  viga debe ser $0$, pues no debería haber defleccion en los puntos de apoyo. Es decir $y$ debe satisfacer lo siguiente

  \begin{equation*}
    \begin{cases}
      y^{(4)}(x) = \frac{q(x)}{EI}\\
      y(0) = 0\\
      y(L) = 0\\
      y''(0) = 0\\
      y''(L) = 0
    \end{cases}
  \end{equation*}
\end{sol}

\begin{sol}
\begin{enumerate}
  \item Verifiquemos que $L(a, b) = k_{1}(k_{2}a - b - k_{3})^{2}$ es convexa.
        Notemos que al ser de clase $\mathcal{C}^{2}$, solo es necesario calcular una de las
        derivadas cruzadas, pues coinciden. Calculemos las derivadas
        \begin{gather*}
          \partial_{a} L(a, b) = 2k_{1}(k_{2}a - b -k_{3}) \cdot k_{2} \implies \partial_{aa}L(a, b) = 2k_{1}k_{2}^{2}\\
          \partial_{b} L(a, b) = -2k_{1}(k_{2}a - b - k_{3}) \implies \partial_{bb} = 2k_{1}\\
          \partial_{ab} L(a, b) = -2k_{1}k_{2}
        \end{gather*}

        Por lo tanto tenemos que el diferencial de $L$ se ve de la siguiente forma
        \begin{equation*}
          DL(a, b) = \begin{bmatrix}
            2k_{1}k_{2}^{2} & -2k_{1}k_{2}\\
            -2k_{1}k_{2}    & 2k_{1}
                     \end{bmatrix}
        \end{equation*}

        Aplicando el criterio de Sylvester tenemos que el primer subdeterminante es $2k_{1}k_{2}^{2} \geq 0$ y
        el determinante de la matriz completa es
        \begin{equation*}
          \det DL(a, b) = 4k_{1}^{2}k_{2}^{2} - 4k_{1}^{2}k_{2}^{2} = 0
        \end{equation*}

        Por lo tanto es una matriz definida positiva, lo que implica que la función es convexa.

  \item El problema no tiene soluci\'on en $C$, consideremos el siguiente problema de valor inicial para
        $\lambda \in \NN$
        \begin{equation*}
          \begin{cases}
            k_{2}y - y' - k_{3} = \lambda\\
            y(0) = y_{0}
          \end{cases}
        \end{equation*}

        Resolveremos este problema de valor inicial por metodos elementales, notemos que es de variables separables
        \begin{equation*}
          \frac{y'}{k_{2}y - k_{3} - \lambda} = 1
        \end{equation*}

        Integrando obtenemos que
        \begin{equation*}
          \frac{1}{k_{2}} \ln(k_{2}y - k_{3} - \lambda) = x + C_{1} \implies \ln(k_{2}y - k_{3} - \lambda) = k_{2}x + k_{2}C_{1}
        \end{equation*}

        tomando exponencial
        \begin{equation*}
          k_{2}y - k_{3} - \lambda = Ck_{2}e^{k_{2}x} \implies y = Ce^{k_{2}x} + \frac{k_{3} + \lambda}{k_{2}}
        \end{equation*}

        Luego para la condición inicial tenemos que
        \begin{equation*}
          y_{0} = C + \frac{k_{3} + \lambda}{k_{2}} \implies y_{0} - \frac{k_{3} + \lambda}{k_{2}} = C
        \end{equation*}

        Se puede verificar que en efecto es una solución de la edo. Definamos $y_{\lambda}$ como la solución al
        problema para alguna lambda. Luego tenemos que $y_{\lambda} \in \mathcal{C}^{\infty}([0, T])$ y
        además $y_{\lambda}(0) = y_{0}$, por lo tanto $y_{\lambda} \in C$. Notemos que
        \begin{equation*}
          J(y_{\lambda}) = \int_{0}^{T} k_{1}\lambda^{2} dx = k_{1} \lambda^{2} T
        \end{equation*}

        Lo cual se va a infinito cuando $\lambda$ se va a infinito. Por lo tanto el problema \textbf{no} tiene solución.
\end{enumerate}
\end{sol}

\end{document}
